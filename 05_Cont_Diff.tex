\documentclass{article}
\usepackage[a4paper, total={6in, 10in}]{geometry}

% Allow the usage of UTF-8 characters
\usepackage[utf8]{inputenc}
% Allow the usage of graphics (.png, .jpg)
\usepackage[T2A]{fontenc}

%Hyphenation rules
%--------------------------------------
\usepackage{hyphenat}
\hyphenation{ма-те-ма-ти-ка вос-ста-нав-ли-вать}
%--------------------------------------
\usepackage[english, russian]{babel}
\usepackage{wrapfig}
\usepackage{graphicx}
%\usepackage{math}
\usepackage{amsmath, amssymb}
\usepackage{lipsum}
% Start the document
\begin{document}

% Create a new 1st level heading
\paragraph{05 Определение дифференцируемости функции. Теорема о связи дифференцируемости с существованием производной.}

\subsubsection*{Определение 1.} 

Функция $y=f(x)$ называтся дифференцируемой в точке $x_0$, если ее приращение $\Delta y$ в этой точке можно записать в виде:

\begin{equation}
	\Delta y = A\cdot\Delta x+\alpha (\Delta x)\cdot\Delta x\text{,}
\end{equation}

$A$ - число и $A$ не зависит от $\Delta x$, $\alpha(\Delta x)$ - бесконечно малая функция при $\Delta x\to 0$.


\subsubsection*{Теорема 1.} {О связи дифференцируемости с существованием производной}

Функция $y=f(x)$ дифференцируема в точке $x_0$ тогда и только тогда, когда существует конечная производная в точке $x_0$.

\subsubsection*{Доказательство необходимости:}

По условию функция $y=f(x)$ дифференцируема в точке $x_0$, докажем, что существует конечная производная $f'(x)$.

По определению дифференцируемой функции в точке:
\[
	\Delta y = A\cdot\Delta x+\alpha(\Delta x)\cdot\Delta x \text{,}
\]
где $\alpha (\Delta x)$ - бесконечно малая функция при $\Delta x \to 0$.

Поделим на $\Delta x $ последнее равенство, $\Delta x \not= 0$ :
\[
	\frac{\Delta y}{\Delta x} = A + \alpha(\Delta x)\text{.}
\]

Перейдем к пределу при $\Delta x\to 0$ :
\begin{equation*}
\begin{aligned}
	& \lim_{\Delta x\to 0}\frac{\Delta y}{\Delta x} = \lim_{\Delta x\to 0} \Bigl( A + \alpha(\Delta x)\Bigl) = 
	\left[ 
	\begin{aligned} 
		&\text{по теореме о} \\ 
		&\text{пределе суммы}
	\end{aligned} \right] =
	\lim_{\Delta x\to 0} A + \lim_{\Delta x\to 0}\alpha(\Delta x) = \\
	& = \left[ 
	\begin{aligned} 
		&\text{$A$ не засивит от $\Delta x$,} \\ 
		&\text{$\alpha(\Delta x)$ - бесконечно малая при $\Delta x\to 0$}
	\end{aligned} \right] = A + 0 = A \text{ (существует и конечен).}
\end{aligned}
\end{equation*}

С другой стороны, по определению производной функции в точке: $\lim_{\Delta x\to 0}\frac{\Delta y}{\Delta x} = f'(x_0)$, значит существует конечная производная %$f'(x_0)$ и
$f'(x_0) = A.$

Равенство (1) примет вид: $\Delta y = f'(x_0)\cdot\Delta x+\alpha (\Delta x)\cdot\Delta x.$

\subsubsection*{Доказательство достаточности:}
Пусть существует конечная $f'(x_0)$.

Докажем, что функция $f(x)$ дифференцируема в точке $x_0$.

По условию существует конечная $f'(x_0)$, значит, существует $\lim_{\Delta x\to 0} \frac{\Delta y}{\Delta x}=f'(x_0).$

По теореме о представлении функции, имеющей конечный предел в точке:
\[
	\lim_{\Delta x\to 0} \frac{\Delta y}{\Delta x} = f'(x_0) + \alpha(\Delta x) \text{,}
\]
где $\alpha(\Delta x)$ - бесконечно малая функция при $\Delta x \to 0$.

Тогда получаем:

\[
\Delta y = f'(x_0)\cdot\Delta x + \alpha(\Delta x)\cdot\Delta x
\]
Равенсто (1) имеет место, значит функция $y=f(x)$ дифференцируема в точке $x_0.$

\end{document}
