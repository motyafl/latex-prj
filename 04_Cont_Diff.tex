\documentclass{article}
\usepackage[a4paper, total={6in, 8in}]{geometry}

% Allow the usage of UTF-8 characters
\usepackage[utf8]{inputenc}
% Allow the usage of graphics (.png, .jpg)
\usepackage[T2A]{fontenc}

%Hyphenation rules
%--------------------------------------
\usepackage{hyphenat}
\hyphenation{ма-те-ма-ти-ка вос-ста-нав-ли-вать}
%--------------------------------------
\usepackage[english, russian]{babel}
\usepackage{wrapfig}
\usepackage{graphicx}
%\usepackage{math}
\usepackage{amsmath, amssymb}
% Start the document
\begin{document}

% Create a new 1st level heading
\paragraph{04 Определение дифференцируемости функции. Теорема о связи непрерывности и дифференцируемости.}

\subsubsection*{Определение 1.}

Функция $y=f(x)$ называется дифференцируемой в точке $x_0$, если ее приращение $\Delta y$ в этой точке можно записать в виде:


\[
	\Delta y = A\cdot\Delta x+\alpha (\Delta x)\cdot\Delta x\text{,}
\]


$A$ - число и $A$ не зависит от $\Delta x$, $\alpha(\Delta x)$ - бесконечно малая функция при $\Delta x\rightarrow 0$.


\subsubsection*{Теорема 2.} {О связи дифференцируемости и непрерывности.}


Если функция $y=f(x)$ дифференцируема в точке $x_0$, то эта функция является непрерывной в точке $x_0$.

Обратное утверждение неверно.

\subsubsection*{Доказательство.}

По условию фунция $y=f(x)$ дифференцируема в точке $x_0$

По определению дифференцируемой функции в точке приращение функции в этой точке:
\begin{equation*}
	\Delta y = A\cdot\Delta x+\alpha(\Delta x)\cdot\Delta x =
	\left[
	\begin{aligned}
		& \text{по теореме о связи дифференцируемости} \\
		& \text{с существованием производной}
	\end{aligned} \right ] = f' (x_0)\cdot\Delta x+\alpha(\Delta x)\cdot\Delta x \text{,}
\end{equation*}

где $\alpha (\Delta x)$ - бесконечно малая функция при $\Delta x \rightarrow 0$.

Найдем $\lim_{\Delta x\rightarrow 0} \Delta y$ :
\begin{equation*}
\begin{aligned}
	\lim_{\Delta x \rightarrow 0} \Delta y & = \lim_{\Delta x \rightarrow 0} (f'(x_0)\cdot\Delta x +\alpha(\Delta x)\cdot\Delta x) = \Bigl[ \text{применим теорему о пределе суммы} \Bigl] = \\
	& = \lim_{\Delta x \rightarrow 0} f'(x_0)\cdot\Delta x + \lim_{\Delta x \rightarrow 0} \alpha(\Delta x)\cdot\Delta x = f'(x_0)\cdot 0 + 0\cdot 0 = 0 \text{.}
\end{aligned}
\end{equation*} 

Получаем, что бесконечно малому приращению аргумента $\Delta x\rightarrow 0$ соответствует бесконечно малое приращение функции $\Delta y\rightarrow 0$.

По эквивалентному определению непрерывной функции: $f(x)$ непрерывна в точке $x_0$.

\begin{wrapfigure}{r}{0.2\textwidth}
%\centering
\includegraphics[width=0.26\textwidth]{images/fun_1.png}
\end{wrapfigure}

Чтобы показать, что обратное утверждение неверно, приведем пример.

\subsubsection*{Пример 1.}


В пример приведем функцию $y=|x|$, $x_0 = 0$.

%\begin{centre}
%    \includegraphics[width=100pt]{images/fun.jpg}
%\end{centre}

Функция $y=|x|$ непрерывна в точке $x_0 = 0$ (см. график)

Посчитаем правую и левую производные функции $y=|x|$ :
\begin{equation*}
\begin{aligned}
	& y'(0+0) = \lim_{ \Delta x\rightarrow 0+0}\frac{\Delta y}{\Delta x} = \Bigl[\Delta x > 0, \text{ т.е. } y = x \Bigl] = \lim_{\Delta x\rightarrow 0+0}\frac{(x+\Delta x)-x}{\Delta x} = \lim_{\Delta x \rightarrow 0} \frac{\Delta x}{\Delta x} = 1,\\
	& y'(0-0) = \lim_{\Delta x\rightarrow 0-0}\frac{\Delta y}{\Delta x} = \Bigl[ \Delta x < 0, \text{ т.е. } y = -x \Bigl] = \lim_{\Delta x\rightarrow 0+0}\frac{-(x+\Delta x)-(-x)}{\Delta x} = \lim_{\Delta x \rightarrow 0} \frac{-\Delta x}{\Delta x} = -1.
\end{aligned}
\end{equation*}

$1\not= -1,$ значит, $\lim_{\Delta x\rightarrow 0} \frac{\Delta y}{\Delta x}$ не существует $\Rightarrow$ по теореме о связи дифференцируемости с существованием производной функция не дифференцируема в точке $x_0$.
\end{document}
